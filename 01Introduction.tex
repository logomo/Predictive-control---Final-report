\chapter{Introduction}\label{ch:introduction}

\noindent This chapter specifies motivation and objectives of research in Model Predictive Control (MPC) and its role in future obstacle avoidance framework. This chapter also summarizes previous research in Model Predictive Control in general field and also in Unmanned Autonomous Vehicles specific field.

\section{Objectives of the report}\noindent
This work concerns the use of a Model Predictive Control scheme to deal with the problem of optimally controlling an Unmanned Autonomous Vehicle \textit{(UAV)} in an certain environment and endowed with the capability of avoiding collision with obstacles. The controlled system consist of conventional time continuous control, which handles all low level task. The continuous part is con rolled by movement automaton $\mathscr{MA}$ which is discrete control mechanism of \textit{open-loop hybrid automaton}. The goal of this report is to propose model predictive control, which will produce movement chain command for movement automaton. 

The development of modern control concepts can be traced back to the work of Kalman in the early 1960s with the linear quadratic regulator (LQR) \cite{kalman1960contributions}. 

In the late 1970s various articles reported successful applications of model predictive control in the industry, principally the ones by Richalet \cite{richalet1978model} presenting Model Predictive Heuristic Control (later known as Model Algorithmic Control (MAC)) and those of Cutler and Ramaker \cite{cutler1980dynamic} with Dynamic Matrix Control (DMC). The  common theme of these strategies was the idea of using a dynamic model of the process (impulse response in the former and step response in the later) to predict the effect of the future control actions, which were determined by minimizing the predicted error subject to operational restrictions. The optimization is repeated at each sampling period with updated information from the process. These formulations were algorithmic and also heuristic and took advantage of the increasing potential of digital computers at the time. Stability was not addressed theoretically and the initial versions of MPC were not automatically stabilizing. However, by focusing on stable plants and choosing a horizon large enough compared to the settling time of the plant, stability is achieved after playing with the weights of the cost function. 

Later on a second generation of MPC such as quadratic dynamic matrix control(QDMC) by Garcia and Morshedi \cite{garcia1986quadratic}, used quadratic programming to solve the constrained open-loop optimal control problem where the system is linear, the cost quadratic, the control and state constraints are defined by linear inequalities. Another line of work arose independently around adaptive control ideas developing strategies essentially for mono-variable processes formulated with transfer function models (for which less parameters are required in the identification of the model) and Diophantine equation was used to calculate future input. The first initiative came with the Minimum Variance Control where the performance index to be minimized is a quadratic function of the error between the most recent output and the reference. In order to deal with non-minimum phase plants a penalized input was placed in the objective function and this became the Generalized Minimum Variance (GVM) control. To overcome the  limitation on the horizon, Peterka \cite{peterka1984predictor} developed the Predictor-Based Self-Tuning Control. Extended Prediction Self Adaptive Control (EPSAC) by De Keyser \cite{de1985extended} proposes a constant control signal starting from the present moment while using a sub-optimal predictor instead of solving a Diophantine equation. Later on the input was replaced by the increment in the control signal to guarantee a zero steady-state error. Based on the ideas of GVM Clarke \cite{clarke1987generalized} developed the Generalized Predictive Control (GPC) and is today one of the most popular methods.  Receding horizon approach was first described by Chen \cite{chen1982receding} for continuous systems. 

Stability for nonlinear continuous systems is discussed in \cite{chen1998nonlinear}. Stability in hybrid systems is problematic, due to changing system dynamic and switching state. Stability and robustness for discrete hybrid systems, similar to movement automaton have been proven in \cite{lazar2007discrete,lazar2006model}. Input to output-stability is other important factor, in case that state is not observable. Positional parameters are derived output parameters instead of vehicle state in most cases. The input-output stability for discrete hybrid systems have been discussed in \cite{lazar2009predictive}.

\section{Motivation}\noindent
Main motivation of this report is to propose Model Predictive Control for Movement automaton. MPC can be used in various tasks from continuous time low level control to event based control on higher levels. Movement automaton is ideal \textit{interface control}, because it separates low level control from event based control and enables to control different systems with same set of event based control implementation. Movement automaton is \textit{command consumer}, therefore some sort of \textit{command producer} need to be introduced. 

MPC is ideal for \textit{command producer}, because it can be used for wide variety of control problems without changing non-parametrized structure. MPC is \textit{modular} in terms of \textit{software engineering}. MPC first implementations were for discrete systems. Input of movement automaton can be transformed discrete system.  Output of Model predictive control can be transformed into \textit{movement chain}. Related literature support this statement.

Tracking of feasible avoidance path or semi-optimal trajectory in general can be addressed by MPC control. Borrelli \cite{borrelli2006mpc} proposed optimal trajectory control low level. However, this work main focus is on high level trajectory prediction and tracking which is executed by movement automaton. Therefore this work can be classified as MPC for hybrid discrete trajectory tracking. Singh used non-linear MPC for trajectory tracking of single UAV in city landscape \cite{singh2001trajectory}. Reactive obstacle avoidance based on MPC hybrid control have been implemented in \cite{shim2007evasive}, this approach is close to movement automaton control, because it used informal representation of movements in event-like environment. However proposed MPC did not addressed continuity of control $u(k)$. Cooperative path plannig for multiple UAV was executed trough MPC control in \cite{wang2007cooperative}. However mentioned approach was event based and did not addressed synchronization of control actions executed by multiple UAV`s.

\section{Goals}\noindent
Main goals of this work is to address following issues, which will contribute to complex obstacle avoidance system:
\begin{enumerate}
    \item\textit{Optimal control mapping from higher to lower level} - mapping of high level commands to lower level continuous signal. This is one of main goal of the this work
    \item\textit{Control architecture for model predictive control control} - provide control framework architecture combining event based, discrete time higher level optimal control and continuous lower level optimal control.
    \item\textit{Predictor mechanism} - develop predictor mechanism for movement chain prediction. This mechanism should not take control and state noise into account.
\end{enumerate}

\section{Organization of the report}\noindent
Report is organized in following manner:
\begin{itemize}
    \item\textit{Chapter \ref{ch:introduction}} presents motivation, goals and objectives of report.
    \item\textit{Chapter \ref{ch:UAVControlProblem}} defines control principle with emphasis to principle of optimality. It also outlines assumptions used for proof of concept.
    \item\textit{Chapter \ref{ch:stateofArt}} provides overview of reach sets, movement automaton and other theoretical supplements used in this work.
    \item\textit{Chapter \ref{ch:problemFormulation3D}} provides problem formulation and it is addressing issues of optimal control.
    \item\textit{Chapter \ref{ch:controlapproach3D}} describes control approach in detail with used techniques with emphasis on optimal control.
    \item\textit{Chapter \ref{ch:controlapproach3D}} addresses control approach on abstract level framework and movement automaton $\mathscr{MA}$ movements $m_i(t_i)$ mapping to continuous signal.
    \item\textit{Chapter \ref{ch:simulationResults}} is focused on simulation results of proposed optimal control approach, it outlines preliminaries of obstacle avoidance framework.
    \item\textit{Chapter \ref{ch:conclusion}} is summarizing model predictive control results and outlines future research heading. 
\end{itemize}